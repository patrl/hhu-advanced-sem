% Created 2023-10-10 Tue 16:02
% Intended LaTeX compiler: pdflatex
\documentclass[letterpaper,parskip=half]{scrartcl}
\usepackage[utf8]{inputenc}
\usepackage[T1]{fontenc}
\usepackage{graphicx}
\usepackage{longtable}
\usepackage{wrapfig}
\usepackage{rotating}
\usepackage[normalem]{ulem}
\usepackage{amsmath}
\usepackage{amssymb}
\usepackage{capt-of}
\usepackage{hyperref}
\usepackage{braket}
\input{boilerplate}
\author{Patrick D. Elliott}
\date{\today}
\title{Semantics in Generative Grammar\\\medskip
\large Notes from Class 1}
\hypersetup{
 pdfauthor={Patrick D. Elliott},
 pdftitle={Semantics in Generative Grammar},
 pdfkeywords={},
 pdfsubject={},
 pdfcreator={Emacs 29.1 (Org mode 9.7-pre)}, 
 pdflang={English}}
\usepackage{biblatex}
\addbibresource{../../bibliography/master.bib}
\addbibresource{~/repos/bibliography/master.bib}
\begin{document}

\maketitle
\tableofcontents

\section{Reading for next week}
\label{sec:org82fcd87}

\begin{itemize}
\item Chapters 1 and 2 of \autocite{HeimKratzer1998}, i.e., up to the end of \textbf{Executing the Fregean Program}.
\end{itemize}
\section{Truth-conditions and compositionality}
\label{sec:org17c8d5a}

\begin{itemize}
\item Conjecture: to know the meaning of a sentence is to know its \emph{truth-conditions} (Alfred Tarski).
\end{itemize}

\begin{exe}
\ex The sentence ``snow is white'' is true if and only if snow is white.
\label{org53ca671}
\end{exe}

\begin{itemize}
\item More general schema (as long as both the object language and the meta-language are the same):
\end{itemize}

\begin{exe}
\ex The sentence ``\(\_\_\_\)'' is true if and only if \(\_\_\_\)
\label{org07779c4}
\end{exe}

\begin{itemize}
\item Is this trivial/circular?
\begin{itemize}
\item One way of understanding this conjecture: semantic competence involves ``knowing'' the associations between particular sentences and ways in which the world can be (as described by the meta-language).
\item Crucially, we know the truth-conditions of sentences we've never heard before.
\item Semantic competence can't amount to just learning associations of sentences-meanings, since the meanings of sentences relate to the meanings of their parts in systematic ways (Gottlob Frege).
\end{itemize}
\item How are meanings put together? Frege's conjecture was that certain meanings are 'incomplete'; in his terms \emph{unsaturated}.
\end{itemize}

\begin{quote}
``Statements in general, just like equations or inequalities or expressions in
Analysis, can be imagined to be split up into two parts; one complete in
itself, and the other in need of supplementation, or ''unsaturated.`` Thus,
e.g., we split up the sentence ''Caesar conquered Gaul`` into '' Caesar`` and ''conquered Gaul.`` The second part is ''unsaturated`` - it contains an empty place; only when this place is filled up with a proper name, or with an expression that replaces a proper name, does a complete sense appear. Here too I give the name ''function`` to what this ''unsaturated`` part stands for. In this case the argument is Caesar.''\\[0pt]
-- Frege
\end{quote}

\begin{itemize}
\item Frege models the notion of an ``unsaturated'' meaning as a function looking for an argument.
\end{itemize}
\section{Formal preliminaries}
\label{sec:org23f6cde}

\subsection{Sets}
\label{sec:orgbe54d9c}

\begin{itemize}
\item A set \(A\) is a collection of objects called \emph{members} or \emph{elements} of the set; \(x\) is an element of \(A\) is written \(x \in A\).
\item Sets may have finitely, or infinitely many members.
\item Sets are identified with their members; there is exactly one set with no members: the \emph{empty set}, written \(\emptyset\).
\end{itemize}
\subsection{Set relations}
\label{sec:orgd63a9c0}

\begin{itemize}
\item Two sets are equal \(A = B\) if \(A\) and \(B\) have the same members.
\begin{itemize}
\item The order in which members are written doesn't matter for set equality.
\end{itemize}
\item \(A\) is a subset of \(B\) \(A \subseteq B\), if \(\forall x \in A\), \(x \in B\).
\item Which of the following statements (if any) are true?
\begin{itemize}
\item For any set \(A\), \(\emptyset \in A\).
\item For any set \(A\), \(\emptyset \subseteq A\).
\item For any set \(A\), \(A \subseteq A\).
\end{itemize}
\begin{itemize}
\item \(A\) is a \emph{strict} subset of \(B\), \(A \subset B\), if \(A \subseteq B\), and \(A \neq B\).
\item \(A\) is a superset of \(B\), \(A \supseteq B\), if \(B \subseteq A\).
\begin{itemize}
\item Strict supersethood \(A \sup B\) has the obvious definition (what is it?).
\end{itemize}
\end{itemize}
\end{itemize}
\subsection{Set operations}
\label{sec:orgb68f4a0}

\begin{itemize}
\item \(x \in A \cap B\) (the intersection of \(A\) and \(B\)) iff \(x \in A\) and \(x \in  B\).
\item \(x \in A \cup B\) (the union of \(A\) and \(B\)) iff \(x \in A\) or \(x \in  B\).
\item \(x \in  B - A\) (the complement of \(A\) in \(B\)) iff \(x \in  B\) and \(x \notin A\).
\end{itemize}
\subsection{Defining sets}
\label{sec:org549ea46}

\begin{itemize}
\item Sets can be defined extensionally by explicitly listing their members:
\end{itemize}

\begin{exe}
\ex \(A \coloneq \set{a,b,c}\)
\label{org84fb968}
\end{exe}

\begin{itemize}
\item Sets can be defined intensionally using abstraction notation:
\end{itemize}

\begin{exe}
\ex \(A := \set{x : x\text{ is a cat} }\)
\label{orga86bbe1}
\end{exe}

\begin{itemize}
\item \(a \in \set{ x | \phi }\) iff \(\phi[x \to a]\) is true.\footnote{\(\phi[x \to a]\) is to be understood as the statement \(\phi\), where all occurrences of the variable \(x\) in \(\phi\) have been replaced with \(a\).}
\end{itemize}
\subsection{Questions}
\label{sec:org6e8a28b}

\begin{itemize}
\item If the \(x\) in \(\set{x : x\text{ is a positive integer less than }7 }\) is a place-holder, why do we need it at all? Why don't we just write \(\set{\_ : \_ \text{is a positive integer less than }7 }\)?
\item Consider the following sets. Which one(s) corresponds to \emph{the set of objects which don't like anything}, and which one corresponds to \emph{the set of objects which nothing likes}?
\begin{itemize}
\item \(\set{x : \set{y : x\text{ likes }y } = \emptyset }\)
\item \(\set{x : \set{y : y\text{ likes }x } = \emptyset }\)
\item \(\set{y : \set{x : x\text{ likes y} } = \emptyset }\)
\end{itemize}
\item Why do we need the variable to the left of the colon? Why can't we just write \(\set{x\text{ is a positive integer less than }7 }\).
\item What does the following mean? \(\set{\text{California} :\text{California is a western state}  }\)
\item What about the following? \(\set{x : \text{California is a western state} }\)
\item Evaluate whether the following is true, and show your reasoning: \(29 \in  \set{x : x \in \set{x : x \neq \emptyset }}\).
\end{itemize}
\subsection{Exercise}
\label{sec:org6250a8e}

In each case, say whether or not the equality holds:

\begin{enumerate}
\item \(\set{a} = \set{b}\)
\item \(\set{x : x = a} = \set{a}\)
\item \(\set{x : x\text{ is green} } = \set{y : y\text{ is green} }\)
\item \(\set{x : x\text{ likes }a } = \set{y : y\text{ likes }b }\)
\item \(\set{x : x \in A} = A\)
\item \(\set{x : x \in  \set{y : y \in  B}} = B\)
\item \(\set{x : \set{y : y\text{ likes }x } = \emptyset } = \set{x : \set{x : x\text{ likes }x } = \emptyset }\)
\end{enumerate}


\printbibliography
\end{document}